\documentclass[12pt]{article}

% Packages
\usepackage{lastpage} 
\usepackage{fancyhdr}	%For header and footer
\usepackage{amsmath, amsthm, amssymb} %For mathematics
\usepackage{graphicx}
\usepackage{hyperref}	%For hyperlinks

% Page setup
\setlength{\topmargin}{-0.4in}
\setlength{\topskip}{0.3in}    % between header and text
\setlength{\textheight}{9.in} % height of main text
\setlength{\textwidth}{6.5in}    % width of text
\setlength{\oddsidemargin}{0in} % odd page left margin
\setlength{\evensidemargin}{0in} % even page left margin
\setlength{\parindent}{0pt}	% Suppress the indent

% Macros
\newcommand*{\bv}[1]{\textbf{#1}} % Write vectors in bold case within an equation environment

% Header and Footer
\pagestyle{fancy}%\fancyhead{}
\fancyhf{}	% Clear the default header and footer
\fancyhead[L]{MAE 3195 \\ Name:}
\fancyhead[R]{Credit Sheet 19 \\ \thepage /\pageref{LastPage}}


% Title
\title{MAE 3195, Credit Sheet 19, Tutorial Lesson 8\\
Beams and Frames}
\date{}

% Body of document
\begin{document}
\maketitle

For this homework, read and work through Lesson 8 in your text \textit{ProMechanica Tutorial}. Before you launch Pro/E, make a new sub-directory in your personal MAE3195 directory titled ``HW19'' and use HW19 as your working directory.\\

Since this is a tutorial homework, each individual must perform this work independently. You are encouraged, however, to discuss the assignment with other students as necessary.\\

It is recommended (but not required) that you study the ``Questions for Review'' and create some of the ``Exercise'' models at the end of the chapter. After completing the lesson, answer the following questions and bring this Credit Sheet with you to class on the due date.

\pagebreak

Solve Exercice 1 at the end of Chapter 8.

\begin{enumerate}
	\item Solve the beam problem with cantilevered boundary conditions at the two ends of the beam. Print a fringe plot of the deflection and bending stress in the beam. Also print graphs of the shear and moment diagrams for the entire length of the beam.\\
What are the reactions (forces or moments) at the two ends and the rolling support of the beam?
	\vspace{2.5in}
	
	\item Resolve the beam problem with pinned boundary conditions at the two ends of the beam. Print a fringe plot of the deflection and bending stress in the beam. Also print graphs of the shear and moment diagrams for the entire length of the beam.\\
What are the reactions (forces or moments) at the two ends and the rolling support of the beam?

\end{enumerate}

\end{document}