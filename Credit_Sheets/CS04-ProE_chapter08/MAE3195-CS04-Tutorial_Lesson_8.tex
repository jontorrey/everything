\documentclass[12pt]{article}

% Packages
\usepackage{lastpage} 
\usepackage{fancyhdr}	%For header and footer
\usepackage{amsmath, amsthm, amssymb} %For mathematics
\usepackage{graphicx}
\usepackage{hyperref}	%For hyperlinks

% Page setup
\setlength{\topmargin}{-0.4in}
\setlength{\topskip}{0.3in}    % between header and text
\setlength{\textheight}{9.in} % height of main text
\setlength{\textwidth}{6.5in}    % width of text
\setlength{\oddsidemargin}{0in} % odd page left margin
\setlength{\evensidemargin}{0in} % even page left margin
\setlength{\parindent}{0pt}	% Suppress the indent

% Macros
\newcommand*{\bv}[1]{\textbf{#1}} % Write vectors in bold case within an equation environment

% Header and Footer
\pagestyle{fancy}%\fancyhead{}
\fancyhf{}	% Clear the default header and footer
\fancyhead[L]{MAE 3195 \\ Name:}
\fancyhead[R]{Credit Sheet 4 \\ \thepage /\pageref{LastPage}}


% Title
\title{MAE 3195, Credit Sheet 4, Tutorial Lesson 8\\
Engineering Drawings}
\date{}

% Body of document
\begin{document}
\maketitle

For this homework, read and work through Lesson 8 in your text \textit{ProEngineer Tutorial}. Before you launch Pro/E, make a new sub-directory in your personal MAE3195 directory titled ``HW4'' and use HW4 as your working directory. When your book asks you to create a new \textit{part} named ``lbrack'', name the part ``***-hw4''. When your book asks you to create a new \textit{drawing} named ``lbrack'', name the drawing ``***-hw4''.\\
In the middle of the lesson, your book will ask you to create a new part that looks like a pulley. You can skip these steps, and use the pulley model provided to you by the instructor. Copy the ``pulley.prt'' file provided by the instructor into your HW4 directory. Open the pulley.prt file in Pro/E and rename it (File$>$Rename $|$ On Disk and In Session) to ``***-pulley''. When your book tells you to create a new drawing of the pulley, name your drawing ``***-pulley''.\\

Since this is a tutorial homework, each individual must perform this work independently. You are encouraged, however, to discuss the assignment with other students as necessary.\\

It is recommended (but not required) that you study the ``Questions for Review'' and create some of the ``Exercise'' models at the end of the chapter. After completing the lesson, answer the following questions and bring this Credit Sheet with you to class on the due date.

\pagebreak

\section*{Review Questions}
\begin{enumerate}
	\item What are the differences between ``shown'' and ``created'' dimensions?
	\vspace{1.25in}
	\item How do you set the orientation of a view on a drawing?
	\vspace{1.25in}
	\item How do you create a text note?
	\vspace{1.25in}
	\item Describe the differences between General View, Projected View, and Detailed View.
	\vspace{1.25in}
	\item Give an example of associativity as it applies to part models and drawings.
\end{enumerate}

\end{document}