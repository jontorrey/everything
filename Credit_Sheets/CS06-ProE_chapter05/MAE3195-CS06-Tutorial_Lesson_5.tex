\documentclass[12pt]{article}

% Packages
\usepackage{lastpage} 
\usepackage{fancyhdr}	%For header and footer
\usepackage{amsmath, amsthm, amssymb} %For mathematics
\usepackage{graphicx}
\usepackage{hyperref}	%For hyperlinks

% Page setup
\setlength{\topmargin}{-0.4in}
\setlength{\topskip}{0.3in}    % between header and text
\setlength{\textheight}{9.in} % height of main text
\setlength{\textwidth}{6.5in}    % width of text
\setlength{\oddsidemargin}{0in} % odd page left margin
\setlength{\evensidemargin}{0in} % even page left margin
\setlength{\parindent}{0pt}	% Suppress the indent

% Macros
\newcommand*{\bv}[1]{\textbf{#1}} % Write vectors in bold case within an equation environment

% Header and Footer
\pagestyle{fancy}%\fancyhead{}
\fancyhf{}	% Clear the default header and footer
\fancyhead[L]{MAE 3195 \\ Name:}
\fancyhead[R]{Credit Sheet 6 \\ \thepage /\pageref{LastPage}}


% Title
\title{MAE 3195, Credit Sheet 6, Tutorial Lesson 5\\
Modeling Utilities}
\date{}

% Body of document
\begin{document}
\maketitle

For this homework, read and work through Lesson 5 in your text \textit{ProEngineer  Tutorial}. Before you launch Pro/E, make a new sub-directory in your personal MAE3195 directory titled ``HW6'' and use HW6 as your working directory. Copy the file ``lesson5.prt.1'' from the CD that accompanied your textbook, and paste it into your HW6 directory. Then use File$>$Open in Pro/E to retrieve ``lesson5.prt.1'' from your HW6 working directory.\\

Since this is a tutorial homework, each individual must perform this work independently. You are encouraged, however, to discuss the assignment with other students as necessary.\\

It is recommended (but not required) that you study the ``Questions for Review'' and create some of the ``Exercise'' models at the end of the chapter. After completing the lesson, answer the following questions and bring this Credit Sheet with you to class on the due date.

\pagebreak

\section*{Review Questions}
\begin{enumerate}
	\item How can you find out the order in which features were created? What is this called?
	\vspace{1.25in}
	\item How can you find which are the parent features of a given feature?
	\vspace{1.25in}
	\item How can you find the references used in creating a feature?
	\vspace{1.25in}
	\item How can you add/remove sketch references when you are in Sketcher?
	\vspace{1.25in}
	\item What is the difference between Edit, Edit References, and Edit Definition?
\end{enumerate}

\end{document}