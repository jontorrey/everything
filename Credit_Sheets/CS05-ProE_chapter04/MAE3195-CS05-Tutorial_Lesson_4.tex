\documentclass[12pt]{article}

% Packages
\usepackage{lastpage} 
\usepackage{fancyhdr}	%For header and footer
\usepackage{amsmath, amsthm, amssymb} %For mathematics
\usepackage{graphicx}
\usepackage{hyperref}	%For hyperlinks

% Page setup
\setlength{\topmargin}{-0.4in}
\setlength{\topskip}{0.3in}    % between header and text
\setlength{\textheight}{9.in} % height of main text
\setlength{\textwidth}{6.5in}    % width of text
\setlength{\oddsidemargin}{0in} % odd page left margin
\setlength{\evensidemargin}{0in} % even page left margin
\setlength{\parindent}{0pt}	% Suppress the indent

% Macros
\newcommand*{\bv}[1]{\textbf{#1}} % Write vectors in bold case within an equation environment

% Header and Footer
\pagestyle{fancy}%\fancyhead{}
\fancyhf{}	% Clear the default header and footer
\fancyhead[L]{MAE 3195 \\ Name:}
\fancyhead[R]{Credit Sheet 5 \\ \thepage /\pageref{LastPage}}


% Title
\title{MAE 3195, Credit Sheet 5, Tutorial Lesson 4\\
Revolved Protrusions, Mirror Copies, Model Analysis}
\date{}

% Body of document
\begin{document}
\maketitle

For this homework, read and work through Lesson 4 in your text \textit{ProEngineer  Tutorial}. Before you launch Pro/E, make a new sub-directory in your personal MAE3195 directory titled ``HW5'' and use HW5 as your working directory. When your book asks you to create a new part named ``guide\_pin'', name the part ``***-hw5''.\\

Since this is a tutorial homework, each individual must perform this work independently. You are encouraged, however, to discuss the assignment with other students as necessary.\\

It is recommended (but not required) that you study the ``Questions for Review'' and create some of the ``Exercise'' models at the end of the chapter. After completing the lesson, answer the following questions and bring this Credit Sheet with you to class on the due date.

\pagebreak

\section*{Review Questions}
\begin{enumerate}
	\item In Sketcher, where are the \textit{Trim} and \textit{Extend} commands? What do they do?
	\vspace{1.25in}
	\item What elements are required to create a revolved protrusion?
	\vspace{1.25in}
	\item What are the options for setting the depth of a blind, both-sides protrusion?
	\vspace{1.25in}
	\item When sketching with Intent Manager, why should you deal with and set up your constraints before setting up the dimensioning scheme? Why do you set the dimensions last?
	\vspace{1.25in}
	\item List three ways that parent/child relationships might be created between two features while making a 2D sketch for a protrusion or cut.
\end{enumerate}

\end{document}