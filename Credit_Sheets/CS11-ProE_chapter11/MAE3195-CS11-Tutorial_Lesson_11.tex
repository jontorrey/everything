\documentclass[12pt]{article}

% Packages
\usepackage{lastpage} 
\usepackage{fancyhdr}	%For header and footer
\usepackage{amsmath, amsthm, amssymb} %For mathematics
\usepackage{graphicx}
\usepackage{hyperref}	%For hyperlinks

% Page setup
\setlength{\topmargin}{-0.4in}
\setlength{\topskip}{0.3in}    % between header and text
\setlength{\textheight}{9.in} % height of main text
\setlength{\textwidth}{6.5in}    % width of text
\setlength{\oddsidemargin}{0in} % odd page left margin
\setlength{\evensidemargin}{0in} % even page left margin
\setlength{\parindent}{0pt}	% Suppress the indent

% Macros
\newcommand*{\bv}[1]{\textbf{#1}} % Write vectors in bold case within an equation environment

% Header and Footer
\pagestyle{fancy}%\fancyhead{}
\fancyhf{}	% Clear the default header and footer
\fancyhead[L]{MAE 3195 \\ Name:}
\fancyhead[R]{Credit Sheet 11 \\ \thepage /\pageref{LastPage}}


% Title
\title{MAE 3195, Credit Sheet 11, Tutorial Lesson 11\\
Sweeps and Blends}
\date{}

% Body of document
\begin{document}
\maketitle

For this homework, read and work through Lesson 11 in your text \textit{ProEngineer Tutorial}. Before you launch Pro/E, make a new sub-directory in your personal MAE3195 directory titled ``HW11'' and use HW11 as your working directory. When the book asks you to make a parts named ``s\_brack'', ``sprinkler'', ``blend1'', and ``blend2'', name them respectively ``***-s\_brack'', ``***-sprinkler'', ``***-blend1'', and ``***-blend2'' instead.\\

Since this is a tutorial homework, each individual must perform this work independently. You are encouraged, however, to discuss the assignment with other students as necessary.\\

It is recommended (but not required) that you study the ``Questions for Review'' and create some of the ``Exercise'' models at the end of the chapter. After completing the lesson, answer the following questions and bring this Credit Sheet with you to class on the due date.

\pagebreak

\section*{Review Questions}
\begin{enumerate}
	\item On the back of this sheet, draw a 3D sketch of an example of each of the following sweeps:
	\begin{enumerate}
		\item closed section, open trajectory,
		\item closed section, closed trajectory, and
		\item open section, closed trajectory.
	\end{enumerate}
	
	\vspace{0.25in}
	\item What problem may arise if the swept section is large and sweep trajectory has a small radius arc in it?
	\vspace{1.25in}
	\item What happens if a sweep trajectory has discontinuities (kinks) in it rather than being composed of smooth tangential transitions?
	\vspace{1.25in}
	\item When creating a blend, what is meant by the ``start point'' of the sketch?
	\vspace{1.25in}
	\item What are the essential common characteristics of all sections in a parallel blend?
\end{enumerate}

\end{document}