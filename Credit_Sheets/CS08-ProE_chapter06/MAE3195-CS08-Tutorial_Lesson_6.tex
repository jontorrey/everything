\documentclass[12pt]{article}

% Packages
\usepackage{lastpage} 
\usepackage{fancyhdr}	%For header and footer
\usepackage{amsmath, amsthm, amssymb} %For mathematics
\usepackage{graphicx}
\usepackage{hyperref}	%For hyperlinks

% Page setup
\setlength{\topmargin}{-0.4in}
\setlength{\topskip}{0.3in}    % between header and text
\setlength{\textheight}{9.in} % height of main text
\setlength{\textwidth}{6.5in}    % width of text
\setlength{\oddsidemargin}{0in} % odd page left margin
\setlength{\evensidemargin}{0in} % even page left margin
\setlength{\parindent}{0pt}	% Suppress the indent

% Macros
\newcommand*{\bv}[1]{\textbf{#1}} % Write vectors in bold case within an equation environment

% Header and Footer
\pagestyle{fancy}%\fancyhead{}
\fancyhf{}	% Clear the default header and footer
\fancyhead[L]{MAE 3195 \\ Name:}
\fancyhead[R]{Credit Sheet 8 \\ \thepage /\pageref{LastPage}}


% Title
\title{MAE 3195, Credit Sheet 8, Tutorial Lesson 6\\
Datum Planes and Sketcher Tools}
\date{}

% Body of document
\begin{document}
\maketitle

For this homework, read and work through Lesson 6 in your text \textit{ProEngineer Tutorial}. Before you launch Pro/E, make a new sub-directory in your personal MAE3195 directory titled ``HW8'' and use HW8 as your working directory. When the book asks you to create a part named ``cutter'', name it ``***-hw8'', where *** indicates your primary initials.\\

Since this is a tutorial homework, each individual must perform this work independently. You are encouraged, however, to discuss the assignment with other students as necessary.\\

It is recommended (but not required) that you study the ``Questions for Review'' and create some of the ``Exercise'' models at the end of the chapter. After completing the lesson, answer the following questions and bring this Credit Sheet with you to class on the due date.

\pagebreak

\section*{Review Questions}
\begin{enumerate}
	\item What are the constraint types for creating datum planes? How are these different for make datums?
	\vspace{1.25in}
	\item What are the constraint types for creating datum axes?
	\vspace{1.25in}
	\item For a symmetric solid protrusion, do you specify the depth in each direction, or the total depth? What about for a symmetric cut?
	\vspace{1.25in}
	\item What is the difference between ``Through All'', ``To Next'', and ``Up to Surf'' when specifying an extruded feature�s depth?
	\vspace{1.25in}
	\item Where and how do make datums show up in the model tree?
\end{enumerate}

\end{document}