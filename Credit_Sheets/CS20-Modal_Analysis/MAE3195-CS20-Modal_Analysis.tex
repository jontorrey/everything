\documentclass[12pt]{article}

% Packages
\usepackage{lastpage} 
\usepackage{fancyhdr}	%For header and footer
\usepackage{amsmath, amsthm, amssymb} %For mathematics
\usepackage{graphicx}
\usepackage{hyperref}	%For hyperlinks

% Page setup
\setlength{\topmargin}{-0.4in}
\setlength{\topskip}{0.3in}    % between header and text
\setlength{\textheight}{9.in} % height of main text
\setlength{\textwidth}{6.5in}    % width of text
\setlength{\oddsidemargin}{0in} % odd page left margin
\setlength{\evensidemargin}{0in} % even page left margin
\setlength{\parindent}{0pt}	% Suppress the indent

% Macros
\newcommand*{\bv}[1]{\textbf{#1}} % Write vectors in bold case within an equation environment

% Header and Footer
\pagestyle{fancy}%\fancyhead{}
\fancyhf{}	% Clear the default header and footer
\fancyhead[L]{MAE 3195 \\ Name:}
\fancyhead[R]{Credit Sheet 20 \\ \thepage /\pageref{LastPage}}


% Title
\title{MAE 3195, Credit Sheet 20, Tutorial Lesson 9\\
Modal Analysis}
\date{}

% Body of document
\begin{document}
\maketitle

For this homework, read and work through the section ``Modal Analysis'' of Lesson 9 in your text \textit{ProMechanica Tutorial}. Before you launch Pro/E, make a new sub-directory in your personal MAE3195 directory titled ``HW20'' and use HW20 as your working directory.

\vspace{.25in} \hrule \vspace{.25in}

Save two copies of the ``***-s\_brack'' part you have created for assignment 11. Name one copy ``***-s\_brack\_alu'', and the other one ***-s\_brack\_steel''.\\

\begin{enumerate}
	\item Assign Aluminum 6061 as the material for the ``***-s\_brack\_alu'' part. Switch to ProMechanica. Constrain all the displacements of the rectangular surface opposite to the sweep feature, and run a multi-pass modal analysis to find the five first modes of the \textbf{constrained} part. Print out in separate plots the results showing the deformed shapes of the five modes.
	\vspace{.25in}
	\item Could you use a symmetry to simplify the model before running the modal analysis? Explain why.
	\vspace{.25in}
	\item Run a multi-pass modal analysis on the \textbf{unconstrained} part to find the first five modes, but select a minimum frequency of 1~Hz to get rid of the rigid modes. Print out in separate plots the results showing the deformed shapes of the five modes.\\
Are there some modes with a frequency close to the constrained case? what can you notice about the rectangular surface on their deformed shape?
	\vspace{.25in}
	\item Assign Steel as the material for the ``***-s\_brack\_steel'' part. Switch to ProMechanica. Constrain all the displacements of the rectangular surface opposite to the sweep feature, and run a multi-pass modal analysis to find the five first modes of the constrained part. Print out in separate plots the results showing the deformed shapes of the five modes.\\
	Compare the natural frequencies of the aluminum part and the steel part. Explain the results (hint: think of a simple one-degree-of-freedom spring-mass system, for which the natural circular frequency is defined as $\omega_n = \sqrt{k/m}$).

\end{enumerate}

\end{document}