\documentclass[12pt]{article}

% Packages
\usepackage{lastpage} 
\usepackage{fancyhdr}	%For header and footer
\usepackage{amsmath, amsthm, amssymb} %For mathematics
\usepackage{graphicx}
\usepackage{hyperref}	%For hyperlinks

% Page setup
\setlength{\topmargin}{-0.4in}
\setlength{\topskip}{0.3in}    % between header and text
\setlength{\textheight}{9.in} % height of main text
\setlength{\textwidth}{6.5in}    % width of text
\setlength{\oddsidemargin}{0in} % odd page left margin
\setlength{\evensidemargin}{0in} % even page left margin
\setlength{\parindent}{0pt}	% Suppress the indent

% Macros
\newcommand*{\bv}[1]{\textbf{#1}} % Write vectors in bold case within an equation environment

% Header and Footer
\pagestyle{fancy}%\fancyhead{}
\fancyhf{}	% Clear the default header and footer
\fancyhead[L]{MAE 3195 \\ Name:}
\fancyhead[R]{Credit Sheet 10 \\ \thepage /\pageref{LastPage}}


% Title
\title{MAE 3195, Credit Sheet 10, Tutorial Lesson 10\\
Assembly Operations}
\date{}

% Body of document
\begin{document}
\maketitle

For this homework, read and work through Lesson 10 in your text \textit{ProEngineer Tutorial}. Before you launch Pro/E, make a new sub-directory in your personal MAE3195 directory titled ``HW10''. Copy and paste all of the files from your HW7 (Tutorial Lesson 9) folder into your HW10 folder � be careful to not get rid of the original copies in your HW7 folder. Launch Pro/E and immediately set your working directly to HW10 as your working directory. Work through the tutorial lesson.\\

Since this is a tutorial homework, each individual must perform this work independently. You are encouraged, however, to discuss the assignment with other students as necessary.\\

It is recommended (but not required) that you study the ``Questions for Review'' and create some of the ``Exercise'' models at the end of the chapter. After completing the lesson, answer the following questions and bring this Credit Sheet with you to class on the due date.

\pagebreak

\section*{Review Questions}
\begin{enumerate}
	\item How can you determine the assembly constraints used for a particular component?
	\vspace{1.25in}
	\item What kind of features can be created as assembly features?
	\vspace{1.25in}
	\item Can you modify individual dimensions of a part while in assembly mode? Is this a permanent modification to the part geometry that also exists in the separate part window?
	\vspace{1.25in}
	\item What does ``Automatic Update'' do when creating a cut through an assembly?
	\vspace{1.25in}
	\item What are the differences between ``Suppress'' and ``Hide''?
\end{enumerate}

\end{document}