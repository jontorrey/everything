\documentclass[12pt]{article}

% Packages
\usepackage{lastpage} 
\usepackage{fancyhdr}	%For header and footer
\usepackage{amsmath, amsthm, amssymb} %For mathematics
\usepackage{graphicx}
\usepackage{hyperref}	%For hyperlinks

% Page setup
\setlength{\topmargin}{-0.4in}
\setlength{\topskip}{0.3in}    % between header and text
\setlength{\textheight}{9.in} % height of main text
\setlength{\textwidth}{6.5in}    % width of text
\setlength{\oddsidemargin}{0in} % odd page left margin
\setlength{\evensidemargin}{0in} % even page left margin
\setlength{\parindent}{0pt}	% Suppress the indent

% Macros
\newcommand*{\bv}[1]{\textbf{#1}} % Write vectors in bold case within an equation environment

% Header and Footer
\pagestyle{fancy}%\fancyhead{}
\fancyhf{}	% Clear the default header and footer
\fancyhead[L]{MAE 3195 \\ Name:}
\fancyhead[R]{Credit Sheet 16 \\ \thepage /\pageref{LastPage}}


% Title
\title{MAE 3195, Credit Sheet 16\\ Surface Modeling / Welded Tubes}
\date{}

% Body of document
\begin{document}
\maketitle

Work through the tutorial on modeling structures made of welded tubes. Name your a-arm part ``***-a\_arm.prt''.\\

After completing the tutorial, answer the following questions and bring this Credit Sheet with you to class on the due date.

\pagebreak
\section*{Review Questions}
\begin{enumerate}
	\item What is the volume of a swept surface?
	\vspace{1.25in}
	\item Describe how datum curves were used as references in building the model of the a-arm.
	\vspace{1.25in}
	\item Describe what happens to the features in the model tree when you merge two surfaces.
	\vspace{1.25in}
	\item On the back of this sheet, draw a tree-structure that illustrates the parent-child relationship of all the features in your model.
	\vspace{.25in}
	\item Is it possible to suppress all of the surfaces and solid features to leave only the datum curves in your model?
\end{enumerate}

\end{document}