\documentclass[12pt]{article}

% Packages
\usepackage{lastpage} 
\usepackage{fancyhdr}	%For header and footer
\usepackage{amsmath, amsthm, amssymb} %For mathematics
\usepackage{graphicx}
\usepackage{hyperref}	%For hyperlinks

% Page setup
\setlength{\topmargin}{-0.4in}
\setlength{\topskip}{0.3in}    % between header and text
\setlength{\textheight}{9.in} % height of main text
\setlength{\textwidth}{6.5in}    % width of text
\setlength{\oddsidemargin}{0in} % odd page left margin
\setlength{\evensidemargin}{0in} % even page left margin
\setlength{\parindent}{0pt}	% Suppress the indent

% Macros
\newcommand*{\bv}[1]{\textbf{#1}} % Write vectors in bold case within an equation environment

% Header and Footer
\pagestyle{fancy}%\fancyhead{}
\fancyhf{}	% Clear the default header and footer
\fancyhead[L]{MAE 3195 \\ Spring 2010}
\fancyhead[R]{Credit Sheet 13 \\ \thepage /\pageref{LastPage}}


% Title
\title{MAE 3195, Credit Sheet 13\\ Modeling Spur Gears with Involute Teeth}
\date{}

% Body of document
\begin{document}
\maketitle

Work through the spur gears tutorial. When you are finished with the tutorial, save copy of the original file to ``***-spur\_gear''. Load the file ``***-spur\_gear''. Modify the parameters/dimensions of the part and add new features to model the spur gear identified as ``YD-16'' on the Boston Gear catalog page (12 diametral pitch, 20 degree pressure angle, 16 teeth).\\

Produce a detailed drawing of your gear that labels the dimensions of the gear and its teeth. The drawing should have three views: 
\begin{enumerate}
	\item Front view of the gear (appears circular),
	\item Side view of gear (appears rectangular), and
	\item Detailed view of the involute gear teeth.
\end{enumerate}
	
Add dimensions to the detailed view of the teeth; you probably want to refer to your undergraduate textbook on machine design to refresh your memory. Print a copy of the drawing, and be prepared to hand it in on the due date.

\end{document}