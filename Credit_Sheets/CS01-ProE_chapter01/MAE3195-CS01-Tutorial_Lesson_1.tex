\documentclass[12pt]{article}

% Packages
\usepackage{lastpage} 
\usepackage{fancyhdr}	%For header and footer
\usepackage{amsmath, amsthm, amssymb} %For mathematics
\usepackage{graphicx}
\usepackage{hyperref}	%For hyperlinks

% Page setup
\setlength{\topmargin}{-0.4in}
\setlength{\topskip}{0.3in}    % between header and text
\setlength{\textheight}{9.in} % height of main text
\setlength{\textwidth}{6.5in}    % width of text
\setlength{\oddsidemargin}{0in} % odd page left margin
\setlength{\evensidemargin}{0in} % even page left margin
\setlength{\parindent}{0pt}	% Suppress the indent

% Macros
\newcommand*{\bv}[1]{\textbf{#1}} % Write vectors in bold case within an equation environment

% Header and Footer
\pagestyle{fancy}%\fancyhead{}
\fancyhf{}	% Clear the default header and footer
\fancyhead[L]{MAE 3195 \\ Name:}
\fancyhead[R]{Credit Sheet 1 \\ \thepage /\pageref{LastPage}}


% Title
\title{MAE 3195, Credit Sheet 1, Tutorial Lesson 1\\
User Interface, View Controls and Model Structure}
\date{}

% Body of document
\begin{document}
\maketitle

For this tutorial homework, work through Lesson 1 in your text \textit{ProEngineer Tutorial}.\\  

You will need files provided by the textbook publisher on the CD-ROM that accompanies your textbook. Insert the CD-ROM that accompanies your textbook. Close the software that automatically launches (hitting Esc at Macromedia screen might work).
\begin{enumerate}
	\item Launch Pro/E by double clicking on the Wildfire icon on the desktop.
	\item Set the working directory to appropriate directory on CD-ROM. Select File$>$Working Directory from the drop down menu. Then browse to select\\
PROEWILDFIRE\textbackslash Supplemental\_Files\textbackslash Educational.
	\item You are now prepared for the activities in Lesson 1.\\
\end{enumerate}	

Since this is a tutorial homework, each individual must perform this work independently. You are encouraged, however, to discuss the assignment with other students as necessary. You are permitted to use your textbook and Pro/E on-line help as necessary.\\

It is recommended that you study the ``Questions for Review'' and create some of the ``Exercise'' models at the end of the chapter. After completing the lesson, answer the following questions and bring this Credit Sheet with you to class on the due date.

\pagebreak

\section*{Review Questions}
\begin{enumerate}
	\item List and summarize how the mouse buttons can be used to pan, spin, and zoom the graphics.
	\vspace{1.25in}
	\item Explain why datum planes have two colors.
	\vspace{1.25in}
	\item What is meant by the term ``preselection''?
	\vspace{1.25in}
	\item What is meant by the term ``regeneration''?
	\vspace{1.25in}
	\item What is the regeneration sequence? How can you determine it for a part?
\end{enumerate}

\end{document}