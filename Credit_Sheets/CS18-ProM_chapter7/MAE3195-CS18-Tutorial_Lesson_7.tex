\documentclass[12pt]{article}

% Packages
\usepackage{lastpage} 
\usepackage{fancyhdr}	%For header and footer
\usepackage{amsmath, amsthm, amssymb} %For mathematics
\usepackage{graphicx}
\usepackage{hyperref}	%For hyperlinks

% Page setup
\setlength{\topmargin}{-0.4in}
\setlength{\topskip}{0.3in}    % between header and text
\setlength{\textheight}{9.in} % height of main text
\setlength{\textwidth}{6.5in}    % width of text
\setlength{\oddsidemargin}{0in} % odd page left margin
\setlength{\evensidemargin}{0in} % even page left margin
\setlength{\parindent}{0pt}	% Suppress the indent

% Macros
\newcommand*{\bv}[1]{\textbf{#1}} % Write vectors in bold case within an equation environment

% Header and Footer
\pagestyle{fancy}%\fancyhead{}
\fancyhf{}	% Clear the default header and footer
\fancyhead[L]{MAE 3195 \\ Name:}
\fancyhead[R]{Credit Sheet 18 \\ \thepage /\pageref{LastPage}}


% Title
\title{MAE 3195, Credit Sheet 18, Tutorial Lesson 7\\
Shell Models}
\date{}

% Body of document
\begin{document}
\maketitle

For this homework, read and work through Lesson 7 in your text \textit{ProMechanica Tutorial}. Before you launch Pro/E, make a new sub-directory in your personal MAE3195 directory titled ``HW18'' and use HW18 as your working directory.\\

Since this is a tutorial homework, each individual must perform this work independently. You are encouraged, however, to discuss the assignment with other students as necessary.\\

It is recommended (but not required) that you study the ``Questions for Review'' and create some of the ``Exercise'' models at the end of the chapter. After completing the lesson, answer the following questions and bring this Credit Sheet with you to class on the due date.

\vspace{.25in} \hrule \vspace{.25in}

Solve Exercice 1 at the end of Chapter 7.

\begin{enumerate}
	\item Print graphs of the strain energy and maximum Von Mises stress as functions of the polynomial order during your multi-pass adaptive solution. Comment on the convergence.
	\vspace{2.in}
	
	\item Print a fringe plot of the Von Mises stress within the model, and include a label identifying the location and value of the maximum Von Mises stress.
	\vspace{.25in}

	\item Print a fringe plot of the Magnitude of displacement within the model, and include a label identifying the location and value of the maximum displacement.

\end{enumerate}

\end{document}