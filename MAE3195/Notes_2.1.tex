\documentclass{homework}
\usepackage{graphicx}
\usepackage{amsmath}
\title{Notes Lecture 2.1}
\author{Jon Torrey}

\begin{document}

\textbf{DFMA}

\begin{enumerate}
  \item ANSI tolerancing
  \item DFMA slides
  \item DFA - guide features
\end{enumerate}

\textbf{FEA basics}

\begin{enumerate}
  \item discritized part
  \item meshing charactersitcs
  \item mesh fineness
  \item a real part is not a FEA part
  \item beware oversimplifications
  \item likely only used for linear systems
\end{enumerate}

\textbf{FEA Process}

\begin{enumerate}
    \item PreProcessing
        \begin{enumerate}
            \item Model clean up - remove small sliver features - rounds/holes/etc. - surpress unneeded features
            \item Meshing
            \item Mesh clean up - well shaped (proportional), smooth transition, no distored elements
            \item Proper shape
        \end{enumerate}
    \item simplify part  

\textbf{Pro/Mechanica}

\begin{enumerate}
    \item Solid model creation
    \item Units
    \item Material
    \item Pro/M
    \item interface (Right hand side)
    \item constraint sets
    \item load sets
    \item autogem features
    \item static analysis
    \item post processing
    \item 
\end{enumerate}

\end{document}
