% This syllabus template was created by:
% Brian R. Hall
% Assistant Professor, Champlain College
% www.brianrhall.net

% Document settings
\documentclass[11pt]{article}
\usepackage[margin=1in]{geometry}
\usepackage[pdftex]{graphicx}
\usepackage{multirow}
\usepackage{setspace}
\pagestyle{plain}
\setlength\parindent{0pt}

\begin{document}

% Course information
\LARGE MAE 3195/3196 Syllabus\\
\LARGE Computer-Aided Engineering of Mechanical Systems \\
\LARGE Spring 2014 \\

% Professor information
\large Jon Torrey \\
\large jtorrey@gwu.edu \\
\large Tompkins 402 \\
\large Office Hours: By appointment \\
\begin{center} The included schedule is tentative and can be changed without notice. \\
\end{center}

% Course details
\textbf {\large \\ Course Description:} This course presents major elements of computer-aided engineering systems: interactive graphics, finite element analysis and design sensitivity and optimization. \\

\textbf{Prerequisite(s):} MAE 4193; concurrent registration: MAE 3196.\\

\textbf{Credit Hours:} 3/1 \\

\textbf{\large Text(s):}
\begin{enumerate}
    \item Toogood, Roger, and Jack E. Zecher. Pro/ENGINEER Wildfire 5.0: Tutorial and Multimedia CD. Ingram, 2009.
    \item Toogood, Roger. Pro/Engineer Wildfire 5.0 Mechanical Tutorial: Structure/Thermal. SDC publications, 2009.
    \item Toogood, Roger. Pro/Engineer Wildfire 5.0 Advanced Tutorial. SDC publications, 2009.
\end{enumerate}

\textbf {\large Course Objectives:} \\
MAE 3195 course learning objectives:
\begin{enumerate}% \itemsep-0.4em
    \item Learn to construct 3D geometric models using parametric, associative, solid modeling; 
    \item Learn to capture design intent in 3D solid models
    \item Learn to use finite element analysis to calculate deformation, stress, or strain in 3D solid models
    \item Learn to demonstrate convergence of finite element solutions
    \item Learn to use sensitivity and optimization techniques in solid modeling and finite element analysis
\end{enumerate}
MAE 3196 course learning objectives: 
\begin{enumerate}
    \item Provide design experience using feature-based, parametric associative solid modeling software
    \item Provide design experience using finite element analysis techniques
\end{enumerate}

\newpage
% I recommend using \newpage here if necessary
\textbf{\large Grade Distribution MAE 3195:} \\
\hspace*{40mm}
\begin{tabular}{ l l }
Credit Sheets & 20\% \\
Exam & 25\% \\
Final Project & 55\% \\
\end{tabular} \\

\textbf{\large Grade Distribution MAE 3196:} \\
\hspace*{40mm}
\begin{tabular}{ l l }
Project 1 & 15\% \\
Project 2 & 20\% \\
Project 3 & 25\% \\
Final Project & 40\% \\
\end{tabular} \\

\textbf{\large Letter Grade Distribution:} \\\\
\hspace*{40mm}
\begin{tabular}{ l l | l l }
\textgreater= 93.00 & A & 73.00 - 76.99 & C \\
90.00 - 92.99 & A-  & 70.00 - 72.99 & C- \\
87.00 - 89.99 & B+  & 67.00 - 69.99 & D+ \\
83.00 - 86.99 & B  & 63.00 - 66.99 & D \\
80.00 - 82.99 & B-  & 60.00 - 62.99 & D- \\
77.00 - 79.99 & C+  & \textgreater 60 & F \\
\end{tabular} \\

\textbf{\large MAE 3195 Homework:}
There will be approximately 13 homework assignments to complete throughout the semester. These assignments will typically be based on lessons from lectures and the Pro/E or Pro/Mechanica textbooks. A Credit Sheet (CS) will be provided for each homework assignment that details all required information that must be submitted for credit. Except when otherwise noted on a credit sheet, it is required that students work independently on the homework assignments. In general, however, discussion among students on the homework is encouraged. Five of the assignments will be randomly selected at the beginning of a class and graded. Therefore, each assignment will be worth an equal portion of your final grade (4\% each). \\

\textbf{\large MAE 3196 Homework:}
Three Pro/E projects will be assigned that build competency in constructing 3D solid models that capture design intent. After completing these first three assigned projects, you will spend the remainder of the semester working on your term project that combines solid modeling with finite element analysis. A formal report will be required for the term project. 

Projects should be turned in on the appropriate due date. As in the case of the homework, an assignment sheet will be provided for each project listing all requirements and the due date for the project. \\

\textbf{\large MAE 3195 Exam:}
The "midterm" will be a take-home exam testing the student on materials related to the first part of the semester (primarily solid modeling and optimization).\\

\textbf{\large Term Project:}
A final project will be assigned during the semester. It will require you to use both 3D solid modeling and finite element analysis. A formal presentation, in the form of a power-point, and report will be used to determine grades. If you are also enrolled in MAE 3196, this project will contribute to both your MAE 3195 and MAE 3196 grades. \\
 
\textbf{\large Contesting a Grade:}
If a student thinks that a grade is unfair, he/she can contest it. The student has to submit a written request (e.g. email) within two weeks after the grade is received (two days for the midterm exam and the final project). Depending on the veracity of the presented arguments, the instructor may adjust the grade by adding or removing points. \\

\textbf{\large Late Submission:}
Assignments are due at the beginning of each period, in coordination with the due date given by the schedule. No credit will be given for assignments turned in late, unless there is good reason or previous coordination and granted permission with instructors.\\

\textbf {\large Course Policies:}
This course’s philosophy on academic honesty is best stated as "be reasonable." The course recognizes that interactions with classmates and others can facilitate mastery of the course’s material. However, there remains a line between enlisting the help of another and submitting the work of another. This policy characterizes both sides of that line.\\

The essence of all work that you submit to this course must be your own. Collaboration on problem sets is not permitted except to the extent that you may ask classmates and others for help so long as that help does not reduce to another doing your work for you. Generally speaking, when asking for help, you may show your code to others, but you may not view theirs, so long as you and they respect this policy’s other constraints. Collaboration on the course’s final project is permitted to the extent prescribed by its specification.\\

Below are rules of thumb that (inexhaustively) characterize acts that the course considers reasonable and not reasonable. If in doubt as to whether some act is reasonable, do not commit it. A detailed code of conduct can be found here: http://www.gwu.edu/~ntegrity/code.html.
\begin{itemize}
	\item \textbf {Reasonable}
		\begin{itemize}
			\item Communicating with classmates about problem sets' problems.
			\item Discussing the course's material with others in order to understand it better.
			\item Helping a classmate identify a bug in his or her model, as by viewing, compiling, or running his or her code, even on your own computer.
			\item Turning to the web or elsewhere for instruction beyond the course’s own, for references, and for solutions to technical difficulties, but not for outright solutions to problem set's problems or your own final project.
			\item Whiteboarding solutions to problem sets with others using diagrams.
			\item Working with (and even paying) a tutor to help you with the course, provided the tutor does not do your work for you.
		\end{itemize}
	\item \textbf {Not Reasonable}
		\begin{itemize}
		    \item Accessing a solution from past students to some problem prior to (re-)submitting your own.
		    \item Asking a classmate to see his or her solution to a problem set’s problem before (re-)submitting your own.
		    \item Failing to cite the origins of code or techniques that you discover outside of the course's own lessons and integrate into your own work, even while respecting this policy's other constraints.
		    \item Paying or offering to pay an individual for work that you may submit as (part of) your own.
		    \item Providing or making available solutions to problem sets to individuals who might take this course in the future.
		    \item Searching for or soliciting outright solutions to problem sets online or elsewhere.
		    \item Viewing another’s solution to a problem set’s problem and basing your own solution on it.
	    \end{itemize}
\end{itemize}


\newpage

% Course Outline
\textbf {\large Tentative Course Outline}:

\begin{table}[h!]
\normalsize % The size of the table text can be changed depending on content. Remove if desired.
\begin{tabular}{ | c | c | }
\hline
\textbf{Week} & \textbf{Content} \\
\hline
Week 1 & \begin{minipage}{.85\textwidth}
\begin{itemize} \itemsep-0.4em
	\vspace{1mm}
	\item Syllabus and Course Schedule; Introduction to 
Parametric Solid Modeling; Creating Simple Objects; Basic Drawings
	\item Reading assignment: WF5 Intro Ch 1, 2, 8
	\vspace{1mm}
\end{itemize}
\end{minipage} \\
\hline
Week 2 & \begin{minipage}{.85\textwidth}
\begin{itemize} \itemsep-0.4em
	\vspace{1mm}
	\item Introduction to Finite Element Method; DFMA
	\item Reading assignment: Mechanica Ch 1, 2, 3
	\vspace{1mm}
\end{itemize}
\end{minipage} \\
\hline %%%%%%%%%%%%%%%%%%%%%%%
Week 3 & \begin{minipage}{.85\textwidth}
\begin{itemize} \itemsep-0.4em
	\vspace{1mm}
	\item Parametric Relations and Constraints; Parent/Child Relationships; Three Dimensional Construction Tools
	\item Reading assignment: WF5 Intro 3, 4
	\vspace{1mm}
\end{itemize}
\end{minipage} \\
\hline %%%%%%%%%%%%%%%%%%%%%%%
Week 4 & \begin{minipage}{.85\textwidth}
\begin{itemize} \itemsep-0.4em
	\vspace{1mm}
	\item Datum Features; 3D Annotation; Symmetrical Features in Designs and 2D Drawings;  Assemblies
	\item Reading assignment:  WF5 Intro 5, 6, 7
	\vspace{1mm}
\end{itemize}
\end{minipage} \\
\hline %%%%%%%%%%%%%%%%%%%%%%%
Week 5 & \begin{minipage}{.85\textwidth}
\begin{itemize} \itemsep-0.4em
	\vspace{1mm}
	\item Assemblies; Advanced Assembly Modeling; Convergence of Finite Element Method; 3D finite element models
	\item Reading assignment:  WF Intro 9, 10; Mechanica 3, 4
	\vspace{1mm}
\end{itemize}
\end{minipage} \\
\hline %%%%%%%%%%%%%%%%%%%%%%%
Week 6 & \begin{minipage}{.85\textwidth}
\begin{itemize} \itemsep-0.4em
	\vspace{1mm}
	\item Convergence of Finite Element Method; 3D finite element models
	\item Reading assignment:  Mechanica 3, 4
	\vspace{1mm}
\end{itemize}
\end{minipage} \\
\hline %%%%%%%%%%%%%%%%%%%%%%%
Week 7 & \begin{minipage}{.85\textwidth}
\begin{itemize} \itemsep-0.4em
	\vspace{1mm}
	\item Advanced Modeling Tools; Pro/E Animation; Optimization
	\item Reading assignment:  WF5 Intro 11 Mechanica 4
	\vspace{1mm}
\end{itemize}
\end{minipage} \\
\hline %%%%%%%%%%%%%%%%%%%%%%%
Week 8 & \begin{minipage}{.85\textwidth}
\begin{itemize} \itemsep-0.4em
	\vspace{1mm}
	\item Optimization; Structural optimization; 
	\item Reading assignment:  Mechanica 4, 5
	\vspace{1mm}
\end{itemize}
\end{minipage} \\
\hline %%%%%%%%%%%%%%%%%%%%%%%
Week 9 & \begin{minipage}{.85\textwidth}
\begin{itemize} \itemsep-0.4em
	\vspace{1mm}
	\item 2D finite element models; Symmetry in Finite Element Models
	\item Reading assignment:  Mechanica 5, 6
	\vspace{1mm}
\end{itemize}
\end{minipage} \\
\hline %%%%%%%%%%%%%%%%%%%%%%%
Week 10 & \begin{minipage}{.85\textwidth}
\begin{itemize} \itemsep-0.4em
	\vspace{1mm}
	\item Shell models; Beams and frames; Modal analysis - Theory
	\item Reading assignment:  Mechanica 7, 8
	\vspace{1mm}
\end{itemize}
\end{minipage} \\
\hline %%%%%%%%%%%%%%%%%%%%%%%
Week 11 & \begin{minipage}{.85\textwidth}
\begin{itemize} \itemsep-0.4em
	\vspace{1mm}
	\item Modal analysis by FEM; Thermal modeling
	\item Reading assignment:  Mechanical 9, 10
	\vspace{1mm}
\end{itemize}
\end{minipage} \\
\hline %%%%%%%%%%%%%%%%%%%%%%%
Week 12 & \begin{minipage}{.85\textwidth}
\begin{itemize} \itemsep-0.4em
	\vspace{1mm}
	\item Miscellaneous topics in FEM
	\item Reading assignment:  Mechanica 9
	\vspace{1mm}
\end{itemize}
\end{minipage} \\
\hline %%%%%%%%%%%%%%%%%%%%%%%
Week 13 & \begin{minipage}{.85\textwidth}
\begin{itemize} \itemsep-0.4em
	\vspace{1mm}
	\item Advanced Modeling Tools
	\item Reading assignment:  
	\vspace{1mm}
\end{itemize}
\end{minipage} \\
\hline %%%%%%%%%%%%%%%%%%%%%%%
Week 14 & \begin{minipage}{.85\textwidth}
\begin{itemize} \itemsep-0.4em
	\vspace{1mm}
	\item Review
	\item Reading assignment:  
	\vspace{1mm}
\end{itemize}
\end{minipage} \\
\hline %%%%%%%%%%%%%%%%%%%%%%%


\end{tabular} 
\end{table}


\end{document}