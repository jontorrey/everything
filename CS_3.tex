\documentclass{homework}
\usepackage{graphicx}
\usepackage{amsmath}
\title{MAE 3195, Credit Sheet 1}
\author{DUE DATE}
\begin{document}

For this homework, read and work through Lesson 3 in your text \textit{ProEngineer Tutorial}. Before you launch Pro/E, make a new sub-directory in your personal MAE3195 directory titled "HW3". Copy the latest revision of "***-hw2.prt" in your HW2 folder, and then paste a copy in your HW3 folder. The latest revision of the "***-hw2.prt" is indicated by the highest suffix in the filename (e.g. rrv-hw2.prt.5 is revision 5 and a later revision than a file named rrv-hw2.prt.4). Be sure not to cut and paste because you want to leave a copy of your work in the HW2 folder. After pasting "***-hw2.prt" into the HW3 folder, rename the file to "***-hw3.prt.*".\\

Launch Pro/E and immediately set/verify that your working directory is set to HW3. When your textbook tells you to retrieve the file from Lesson 2, be sure to select your "***-hw3.prt" file that is now stored in HW3. If you correctly set your working directly to HW3 folder, then HW3 folder should be the default folder after selecting File$>$Open. Proceed with the remainder of Lesson 3 as described in the textbook.\\

Since this is a tutorial homework, each individual must perform this work independently. You are encouraged, however, to discuss the assignment with other students as necessary.\\

It is recommended (but not required) that you study the "Questions for Review" and create some of the "Exercise" models at the end of the chapter. After completing the lesson, answer the following questions and bring this Credit Sheet with you to class on the due date.

\problem{} What is the difference between the terms "protrusion" and "extrusion"?
\solution

\problem{} Once a feature has been created, describe two methods for changing its dimension.
\solution

\problem{} Describe what is meant by design intent.
\solution

\problem{} Describe the various techniques for capturing design intent within a 2D sketch.
\solution

\problem{} 
\solutions
\end{document}
